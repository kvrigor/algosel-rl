\chapter{Introduction}
\section{Introduction}
The \textit{algorithm selection} (AS) problem presents the question, "Given a set of algorithms, how to choose one that best solves a given problem?" AS is best applied in solving intractable problems, whose solutions often require brute-force or heuristic approaches. An effective method of performing AS has broad implications on solving real-world problems which often are intractable in nature.

\section{Problem Background}
A survey of AS techniques \citep{kotthoff2016algorithm} reported a majority of approaches that used supervised learning in training an AS model. Can reinforcement learning (RL) techniques be used? Recently, RL emerged as a superior approach to learning as demonstrated in applications such as automatic design of neural network architecture \citep{zoph2016neural} and board game AI exhibiting superhuman performance \citep{silver2016mastering}. This study explores the potential of RL in training AS models with superior performance. The AS models in this study focus on solving a set of hard computational tasks called subgraph isomorphism problems. This is an NP-complete problem under graph theory where the objective is to determine if a graph contains a pattern of a much smaller graph. A recent study tackled this problem by building and training an AS model using pairwise random forest regression \citep{kotthoff2016portfolios}. This study tries to improve upon the results of the previous study by proposing RL as an alternative to training AS models.

\section{Problem Statement}
This study addresses how reinforcement learning can be applied to algorithm selection for subgraph isomorphism problems. Specifically, it attempts to answer the following questions:
\begin{itemize}
	\item How can the algorithm selection problem be viewed from the perspective of reinforcement learning?
	\item Can reinforcement learning be used to train an algorithm selection model in solving subgraph isomorphism problems?
	\item Does an algorithm selection model perform better when trained using reinforcement learning as compared to supervised learning?
\end{itemize}

\section{Research Objectives}
This study aims to use reinforcement learning in training an algorithm selection model for subgraph isomorphism problems. Specifically, it attempts to meet the following objectives:
\begin{itemize}
	\item To prepare the subgraph isomorphism dataset for training the algorithm selection model.
	\item To implement a reinforcement learning algorithm for training an algorithm selection model.
	\item To evaluate whether reinforcement learning can improve the performance of the algorithm selection model as compared to supervised learning.
\end{itemize}

\section{Scope of the Study}
The constraints observed in this study are outlined below:
\begin{itemize}
	\item The AS models are trained and tested using the subgraph isomorphism dataset generated from a previous related study by Kotthoff \citep{kotthoff2016portfolios}
	\item The characteristics of the dataset impose restrictions on how training and testing can proceed.
	\begin{itemize}
		\item The problems focus only on subgraph isomorphism problems.
		\item The set of algorithms are predefined and fixed.
		\item The model can only be trained to match one algorithm to one problem at a time.
	\end{itemize}
	\item Only one RL algorithm is considered in this study.
\end{itemize}

\section{Significance of the Study}
This study contributes to the following:

\begin{itemize}
	\item Investigates the feasibility of training algorithm selection models using reinforcement learning.
	\item Provides insights on how reinforcement learning algorithms can be applied to algorithm selection.
	\item Introduces the potential of reinforcement learning in training effective algorithm selection models.
\end{itemize}
